\documentclass[12pt]{article}

\usepackage[letterpaper,margin=1in]{geometry}

\setlength{\parindent}{0pt}

\usepackage{amssymb}
\usepackage{amsmath}

\newcommand{\assessmentTitle}{
  Standard Assessment 2
}

\usepackage{fancyhdr}
\pagestyle{fancy}
\renewcommand{\headrulewidth}{0pt}% Default \headrulewidth is 0.4pt
\renewcommand{\footrulewidth}{0pt}% Default \footrulewidth is 0pt
\chead{\footnotesize\bf\assessmentTitle}
\cfoot{\small Page \thepage}

\newcommand{\makeHeader}[4]{
\thispagestyle{empty}
\begin{center}
\fbox{\fbox{\parbox{5.5in}{\centering
#1 | #2 | #3 | #4
}}}
\end{center}
\vspace{0.1in}
\makebox[\textwidth]{
  Name:\enspace\hrulefill\hrulefill\hrulefill
}
}

\usepackage{xcolor}

\newcommand{\standardQuestion}[2]{
\newpage
\begin{center}
  \begin{tabular}{|l|c|}
  \hline
    \parbox{5in}{
      \textbf{#1}: This student is able to...\\
      #2
    }
  &
    \parbox{1in}{
      \vspace{0.1in}
      \footnotesize \textcolor{gray}{Mark:}
      \vspace{0.7in}

      \tiny \textcolor{gray}{(Instructor Use Only)}
    }
  \\\hline
  \end{tabular}
\end{center}
}

\newcommand{\csch}{\operatorname{csch}}
\newcommand{\sech}{\operatorname{sech}}



\begin{document}

\makeHeader{
MA 126 }{ Spring 2017 }{ Prof. Clontz }{ \assessmentTitle
}

\begin{itemize}
  \item Each question is prefaced with a Standard for this course.
  \item When grading, each response will be marked as follows:
  \begin{itemize}
    \item \(\checkmark\):
      The response is demonstrates complete understanding of the Standard.
    \item \(\star\):
      The response may indicate full understanding of the Standard, but
      clarification or minor corrections are required.
    \item \(\times\):
      The response does not demonstrate complete understanding of the Standard.
  \end{itemize}
  \item Only responses marked with a \(\checkmark\) mark count toward your
    grade for the semester.
    Visit the course website for more information on how to improve
    \(\star\) and \(\times\) marks.
  \item This Assessment is due after 50 minutes. All blank responses will
    be marked with \(\times\).
\end{itemize}





\standardQuestion{C01}{
  Derive properties of the logarithmic and exponential functions from their definitions.
}

Show that \(5^2=25\) follows from the definition \(a^x = \exp(x\ln(a))\).
(Hint: Use the fact that \(\ln(z)+\ln(z)=\ln(z\times z)\).)





\standardQuestion{C02}{
  Prove hyperbolic function identities.
}

Use the definition
\[
  % \sinh(x) = \frac{e^x-e^{-x}}{2},
  \cosh(x) = \frac{e^x+e^{-x}}{2}
\]
to prove the following identity.
\[
  \cosh(2x) = 2\cosh^2(x)-1
\]



\standardQuestion{C03}{
  Use integration by substitution.
}

Show that \(\int_2^3 x\sqrt{x-2}\,dx=\frac{26}{15}\).




\standardQuestion{S01}{
  Find derivatives and integrals involving logrithmic and exponential functions.
}

a) Find \(\frac{d}{dz}[\ln(3e^z)]\).

\vfill

b) Find \(\displaystyle\int\left(2e+\frac{3}{y}\right)\,dy\).

\vfill



\standardQuestion{S02}{
  Find derivatives and integrals involving hypberbolic functions.
}

a) Find \(\frac{d}{dv}[4\tanh(3v)-\sinh(v^2)]\).

\vfill

b) Find \(\displaystyle\int(\cosh(x)+2\sinh(x))\,dx\).

\vfill



\standardQuestion{S03}{
  Integrate products of trigonometric functions.
}

Find \(\int\sin^3(\theta)\cos^3(\theta)\,d\theta\).



\standardQuestion{S04}{
  Use trigonometric substitution.
}

Find \(\int\frac{2}{1+4x^2}\,dx\).




\standardQuestion{S05}{
  Use partial fractions to integrate rational functions.
}

a) Complete the following partial fraction expansion:

\[
  \frac{f(x)}{(x+3)^3(x^2+7)^2}
    =
  \frac{A}{\hspace{3em}}
    +
  \frac{B}{\hspace{3em}}
    +
  \frac{C}{\hspace{3em}}
    +
  \frac{Dx+E}{\hspace{3em}}
    +
  \frac{Fx+G}{\hspace{3em}}
\]

\vspace{1em}

(Assume the degree of \(f\) is less than \(7\).
You do NOT need to solve for \(A\) through \(G\).)

\vspace{1em}

b) Find \(\int\frac{8x^2-6x+14}{(x-1)(x^2+7)}\,dx\).

\vfill







\newpage

\textcolor{gray}{Use this space if you need extra room for a problem:}

\end{document}
