\documentclass{article}[12pt]

\usepackage{course-notes}
\usepackage{course-notes-rats}

\begin{document}
\begin{module}{E}{Solving Systems of Linear Equations}
% IF-AT Form B003 1-10
\begin{readinessAssuranceTest}

%1 B
\item Which of the following describe the set of all points on the line \(2x+3y=0\)?
\begin{multicols}{4}
\begin{readinessAssuranceTestChoices}
\item \(\displaystyle \left\{ (x,y) \right\}\)
\item \(\displaystyle \left\{ (x,y) \,\middle|\, 2x+3y=0 \right\}\) %Correct
\item \(\displaystyle \left\{ (2x,3y) \right\}\)
\item \(\displaystyle \left\{ (2x,3y) \,\middle|\, 2x+3y=0 \right\}\)
\end{readinessAssuranceTestChoices}
\end{multicols}


%2  D
\item How many solutions are there for the system of linear equations
      represented by the following graph?
    \begin{center}
      \systemWithOneSolutionB[0.23]
    \end{center}

\begin{multicols}{4}
\begin{readinessAssuranceTestChoices}
\item Zero
\item Two
\item Infinitely-many
\item One % correct
\end{readinessAssuranceTestChoices}
\end{multicols}

%3 D
\item Which of the following points is an element of the set 
		\(\displaystyle \left\{ (x,y) \,\middle|\, 3x+4y=12 \right\}\) ?
\begin{multicols}{4}
\begin{readinessAssuranceTestChoices}
\item \( (1,1) \)
\item \( (3,4) \)
\item \( (4,-3) \)
\item \( (8,-3) \) %Correct
\end{readinessAssuranceTestChoices}
\end{multicols}

%4  A
\item How many solutions are there for the system of linear equations
      represented by the following graph? (This graph represents two completely
      overlapping lines.)
    \begin{center}
      \systemWithInfinitelyManySolutions
    \end{center}

\begin{multicols}{4}
\begin{readinessAssuranceTestChoices}
\item Infinitely-many % correct
\item One
\item Two
\item Zero
\end{readinessAssuranceTestChoices}
\end{multicols}


%5  B
\item How many solutions are there for the system of linear equations
      represented by the following graph?
    \begin{center}
      \systemWithOneSolutionA
    \end{center}

\begin{multicols}{4}
\begin{readinessAssuranceTestChoices}
\item Zero
\item One % correct
\item Two
\item Infinitely-many
\end{readinessAssuranceTestChoices}
\end{multicols}


%6  B
\item How many solutions are there for the system of linear equations
      represented by the following graph? (This graph represents two
      non-overlapping parallel lines.)
    \begin{center}
      \systemWithNoSolutions
    \end{center}

\begin{multicols}{4}
\begin{readinessAssuranceTestChoices}
\item Infinitely-many
\item Zero % correct
\item One
\item Two
\end{readinessAssuranceTestChoices}
\end{multicols}


%7  C
\item Solve the following system of linear equations.
      \begin{align*}
      y   &=   2x+5 \\
      y  &=  -x+2
      \end{align*}

\begin{multicols}{4}
\begin{readinessAssuranceTestChoices}
\item There are no solutions.
\item There are infinitely-many solutions.
\item \((x,y)=(-1,3)\) % correct
\item \((x,y)=(4,-2)\)
\end{readinessAssuranceTestChoices}
\end{multicols}


%8  C
\item Solve the following system of linear equations.
      \begin{align*}
      y   &=  3x+5 \\
      y  &=  3x+2
      \end{align*}

\begin{multicols}{4}
\begin{readinessAssuranceTestChoices}
\item There are infinitely-many solutions.
\item
\((x,y)=(3,4)\)
\item There are no solutions. % correct
\item
\((x,y)=(-5,1)\)
\end{readinessAssuranceTestChoices}
\end{multicols}


%9  D
\item Solve the following system of linear equations.
      \begin{align*}
      x+2y   &=   4 \\
      2x-3y  &=  1
      \end{align*}

\begin{multicols}{4}
\begin{readinessAssuranceTestChoices}
\item There are no solutions.
\item There are infinitely-many solutions.
\item
\((x,y)=(-1,4)\)
\item
\((x,y)=(2,1)\) % correct
\end{readinessAssuranceTestChoices}
\end{multicols}


%10 D
\item Solve the following system of linear equations.
      \begin{align*}
      4x-8y   &= 12 \\
      -6x+12y  &=  -18
      \end{align*}

\begin{multicols}{4}
\begin{readinessAssuranceTestChoices}
\item
\((x,y)=(3,3)\)
\item
\((x,y)=(-2,1)\)
\item There are no solutions.
\item There are infinitely-many solutions. % correct
\end{readinessAssuranceTestChoices}
\end{multicols}

\end{readinessAssuranceTest}
\end{module}
\end{document}
