\documentclass[12pt]{exam}

\usepackage{amsmath,amsfonts, amsthm}
\newcommand{\ds}{\ensuremath{\displaystyle}}
\newcommand{\setB}[2]{\left\{#1:#2\right\}}

\printanswers

\begin{document}

\begin{center}
\fbox{\fbox{\parbox{5.5in}{\centering
MA437 - 2019 Spring - Take Home Assessment II - Dr. Clontz
}}}
\end{center}
\vspace{0.1in}
\makebox[\textwidth]{
  Name:\enspace\hrulefill\hrulefill\hrulefill\space
}

\vspace{12pt}

Print and complete this assessment. It is due at the beginning of class on
Thursday February 21.

\vspace{1em}

Circle the most appropriate response for each.

\section*{Chapter 1 Computation}

\begin{questions}

\setcounter{question}{0}

\question Compute \((5,-3)(2,-1)\). 
\begin{choices}
\choice \((5,-9)\) 
\choice \((6,-11)\) 
\choice \((-4,7)\) 
\choice None of these. 
\end{choices}

%\question Let \(z=3-4i\). Find \(z^{-1}\).
%\begin{choices}
%\choice \(\frac{3}{25}+\frac{4}{25}i\)
%\choice \(-\frac{3}{25}+\frac{4}{25}i\)
%\choice \(\frac{4}{25}-\frac{3}{25}i\)
%\choice None of these. 
%\end{choices}

\question Simpify \(\overline{\sqrt3 - 6i - i(5+i\sqrt3)}\). 
\begin{choices}
\choice \(-\sqrt3 +4i\)
\choice \(5\sqrt3 -2+i\)
\choice \(2\sqrt3 +11i \)
\choice None of these. 
\end{choices}

\question Compute \(|10-24i|\)
\begin{choices}
\choice \(-14\)
\choice \(26\)
\choice \(34\)
\choice None of these. 
\end{choices}

%\question Find the exponential form \(re^{i\theta}\) of \(z=2-2i\sqrt 3\).
%\begin{choices}
%\choice \(4e^{-i\pi/3}\)
%\choice \(6e^{-i\pi/6}\)
%\choice \(8e^{i\pi/4}\)
%\choice None of these. 
%\end{choices}

\question Express \((-16)^{-4}\) in \(x+iy\) form.
\begin{choices}
\choice \(\sqrt 2-\sqrt 2 i\)
\choice \(2\sqrt 2+2\sqrt 2 i\)
\choice \(-16\sqrt 2-16\sqrt 2 i\)
\choice None of these. 
\end{choices}

\question If \(z=7e^{-i\pi/6}\), then what is its \textit{principle} argument
\(\operatorname{Arg}(z)\)?
\begin{choices}
\choice \(-7\pi/6\) 
\choice \(-\pi/6\) 
\choice \(7\pi/6\) 
\choice None of these. 
\end{choices}
\end{questions}

%\section*{Chapter 1 Knowledge}
%
%Circle the most appropriate response for each.
%
%\begin{questions}
%
%\setcounter{question}{5}
%
%\question \(\overline{zw}=\overline{z}+\overline{w}\). 
%\begin{choices}
%\choice True
%\choice False
%\end{choices}
%
%\question The points of the set \(\{z\in\mathbb C:|z-i|=4\}\) form a circle. 
%\begin{choices}
%\choice True
%\choice False
%\end{choices}
%
%\question Every complex number may be written in the form \(r\cos\theta+ir\sin\theta\). 
%\begin{choices}
%\choice True
%\choice False
%\end{choices}
%
%\question \(e^{\theta_1}e^{\theta_2}=e^{\theta_1\theta_2}\). 
%\begin{choices}
%\choice True
%\choice False
%\end{choices}
%
%\question The equation \(z^3=4-3i\) has a unique solution. 
%\begin{choices}
%\choice True
%\choice False
%\end{choices}
%\end{questions}

\section*{Chapter 1 Proofs}

Choose at most one of the following exercises to submit to the instructor.

\begin{questions}

\setcounter{question}{10}

\question Prove that \(\frac{-b\pm i\sqrt{4ac-b^2}}{2a}\) is a solution
to \(az^2+bz+c=0\) whenever \(4ac-b^2\geq 0\).
\question Give a formula for the multiplicative inverse \(z^{-1}\) of \(z=re^{i\theta}\)
and prove that it works. 

\end{questions}

\end{document}
