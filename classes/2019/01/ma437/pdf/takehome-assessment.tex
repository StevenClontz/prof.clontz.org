\documentclass[13pt]{exam}

\usepackage{amsmath,amsfonts, amsthm}
\newcommand{\ds}{\ensuremath{\displaystyle}}
\newcommand{\setB}[2]{\left\{#1:#2\right\}}

\printanswers

\begin{document}

\begin{center}
\fbox{\fbox{\parbox{5.5in}{\centering
MA437 - 2019 Spring - Take Home Assessment - Dr. Clontz
}}}
\end{center}
\vspace{0.1in}
\makebox[\textwidth]{
  Name:\enspace\hrulefill\hrulefill\hrulefill\space
}

\vspace{12pt}

Print and complete this assessment. It is due at the beginning of class on
Tuesday February 5.

\vspace{1em}

Circle the most appropriate response for each.

\section*{Module 1 Computation}

\begin{questions}

\setcounter{question}{0}

%\question Compute \((2,-3)(4,-2)\). 
%\begin{choices}
%\choice \(14\) 
%\choice \(-5 \)
%\choice \(-7 \)
%\choice None of these. 
%\end{choices}

\question Let \(z=3-4i\). Find \(z^{-1}\).
\begin{choices}
\choice \(\frac{3}{25}+\frac{4}{25}i\)
\choice \(-\frac{3}{25}+\frac{4}{25}i\)
\choice \(\frac{4}{25}-\frac{3}{25}i\)
\choice None of these. 
\end{choices}

\question Simpify \(\overline{\sqrt2 - 3i - i(5-i\sqrt2)}\). 
\begin{choices}
\choice \(-5\sqrt2 +3i\)
\choice \(7\sqrt2 -1+i\)
\choice \(2\sqrt2 -2i \)
\choice None of these. 
\end{choices}

%\question Compute \(|5-12i|\)
%\begin{choices}
%\choice \(-7\)
%\choice \(13\)
%\choice \(17\)
%\choice None of these. 
%\end{choices}

\question Find the exponential form \(re^{i\theta}\) of \(z=2-2i\sqrt 3\).
\begin{choices}
\choice \(4e^{-i\pi/3}\)
\choice \(6e^{-i\pi/6}\)
\choice \(8e^{i\pi/4}\)
\choice None of these. 
\end{choices}

\question Express \((-\sqrt3+i)^{-3}\) in \(x+iy\) form.
\begin{choices}
\choice \(-\frac{1}{8}i\)
\choice \(\frac{1}{4}-\frac{1}{32}i\)
\choice \(-\frac{1}{16}\)
\choice None of these. 
\end{choices}

\question If \(z=e^{5i\pi/4}\), then what is its \textit{principle} argument
\(\operatorname{Arg}(z)\)?
\begin{choices}
\choice \(-7\pi/4\) 
\choice \(-3\pi/4\) 
\choice \(5\pi/4\) 
\choice None of these. 
\end{choices}
\end{questions}

\section*{Module 1 Knowledge}

Circle the most appropriate response for each.

\begin{questions}

\setcounter{question}{5}

\question \(\overline{zw}=\overline{z}+\overline{w}\). 
\begin{choices}
\choice True
\choice False
\end{choices}

\question The points of the set \(\{z\in\mathbb C:|z-i|=4\}\) form a circle. 
\begin{choices}
\choice True
\choice False
\end{choices}

\question Every complex number may be written in the form \(r\cos\theta+ir\sin\theta\). 
\begin{choices}
\choice True
\choice False
\end{choices}

\question \(e^{\theta_1}e^{\theta_2}=e^{\theta_1\theta_2}\). 
\begin{choices}
\choice True
\choice False
\end{choices}

\question The equation \(z^3=4-3i\) has a unique solution. 
\begin{choices}
\choice True
\choice False
\end{choices}
\end{questions}

\section*{Module 1 Proofs}

Choose at most one of the following exercises to submit to the instructor
on a separate page.

\begin{questions}

\setcounter{question}{10}

\question The statement \(Im(z)=\frac{z+\overline{z}}{2i}\) is false. Fix the formula
  and prove that your fixed formula is true. 
\question Let \(w,z\) be nonzero complex numbers. Prove that the multiplicative inverse of
  \(wz\) is the product of the multiplicative inverse of \(w\) with the multiplicative
  inverse of \(z\).
\question Prove that for any value of \(arg(\frac{3}{z})\), there exists an equal value
  for \(-arg(z)\).

\end{questions}

\end{document}
