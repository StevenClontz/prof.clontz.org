\documentclass[13pt]{exam}

\usepackage{amsmath,amsfonts, amsthm}
\newcommand{\ds}{\ensuremath{\displaystyle}}
\newcommand{\setB}[2]{\left\{#1:#2\right\}}

\printanswers

\begin{document}

\begin{center}
\fbox{\fbox{\parbox{5.5in}{\centering
MA437 - 2019 Spring - Assessment 02/19 - Dr. Clontz
}}}
\end{center}
\vspace{0.1in}
\makebox[\textwidth]{
  Name:\enspace\hrulefill\hrulefill\hrulefill\space
}

\vspace{12pt}

\vspace{1em}

Circle the most appropriate response for each.

\section*{Chapter 2 Computation}

\begin{questions}

\setcounter{question}{0}

\question Let \(f(z)=z^2+\overline{z}\). Find \(f(3-4i)\).
\begin{choices}
\choice \(20+9i\)
\choice \(-7-20i\)
\choice \(-10+9i\)
\choice None of these. 
\end{choices}

\question Find \(\displaystyle\lim_{z\to i-1}\frac{z^2+2z+2}{z+1-i}\). 
\begin{choices}
\choice \(0\)
\choice \(2i\)
\choice \(4-3i\)
\choice None of these. 
\end{choices}

\question Find the value of \(\frac{d}{dz}\left[\frac{z^2+2z+2}{z+1-i}\right]\)
when \(z=i\). 
\begin{choices}
\choice \(1\)
\choice \(-i\)
\choice \(2-2i\)
\choice None of these. 
\end{choices}

\question Let \(z=x+iy\) and \(f(z)=x^2-y^2+(2xy)i\). Find \(f'(1+i)\). 
\begin{choices}
\choice \(3\)
\choice \(-4i\)
\choice \(2+2i\)
\choice None of these. 
\end{choices}

\question Let \(z=x+iy\) and \(f(z)=u(z)+iv(z)=ye^{2ix}\). Find \(v_x\).
\begin{choices}
\choice \(2y\cos(2x)\)
\choice \(2x\sin(2y)\)
\choice \(2e^{xy}\)
\choice None of these. 
\end{choices}
\end{questions}

\section*{Chapter 2 Knowledge}

Circle the most appropriate response for each.

\begin{questions}

\setcounter{question}{5}

\question If \(\lim_{z\to z_0}\frac{f(z)-f(z_0)}{z-z_0}\) exists,
then \(f\) is differentiable at \(z\). 
\begin{choices}
\choice True
\choice False
\end{choices}

\question There exists a continuous function \(f\) such that \(g\) defined by 
\(g(z)=-5f(z)+f(z)^2\) is discontinuous at some point. 
\begin{choices}
\choice True
\choice False
\end{choices}

\question There exists a differentiable function \(f\) such that
every directional derivative of \(f\) exists.
\begin{choices}
\choice True
\choice False
\end{choices}

\question The domain of \(f(z)=\frac{2}{z+\overline{z}}\) is all complex numbers
with a nonzero real part. 
\begin{choices}
\choice True
\choice False
\end{choices}

\question If \(f\) is singular at a point \(z_0\), then there exists a disk 
\(\{z:|z-z_0|<\epsilon\}\) around that point
such that \(f\) is non-differentiable at every point in that disk.
(Put another way, \(z_0\) is far away from any points at which \(f\) is
differentiable.) 
\begin{choices}
\choice True
\choice False
\end{choices}
\end{questions}

\section*{Chapter 2 Proofs}

Prove one (not both) of the following.

\begin{questions}

\setcounter{question}{10}

\question 
Prove that \(\lim_{z\to 1}\frac{z-1}{\overline{z}-1}\) does not exist.
\question 
Prove that \(f(z)=\overline{z}^2\) is differentiable at exactly one point.

\end{questions}

\end{document}
