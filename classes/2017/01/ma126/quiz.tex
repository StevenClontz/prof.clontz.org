\documentclass[12pt]{exam}

\newcommand{\ds}{\ensuremath{\displaystyle}}

\newcommand{\questionDate}[1]{\textcolor{gray}{(#1)}}

\usepackage{amsmath,amsfonts, amsthm}
\usepackage{multicol}
\usepackage{multirow}
\usepackage{harpoon}
\renewcommand{\arraystretch}{1.5}

\newcommand{\harpvec}[1]{\overrightharp{\ensuremath{\mathbf{#1}}}}
\newcommand{\vect}[1]{\ensuremath{\mathbf{#1}}}
\newcommand{\<}{\langle}
\renewcommand{\>}{\rangle}
\newcommand{\p}{\partial}

% ref: http://pgfplots.sourceforge.net/gallery.html
% ref: http://tex.stackexchange.com/a/74575/79754
\usepackage{pgfplots}% This uses tikz
\pgfplotsset{compat=newest}% use newest version
\tikzset{LineStyle/.style={smooth, ultra thick, samples=400}}

% \printanswers

\begin{document}

\begin{center}
\fbox{\fbox{\parbox{5.5in}{\centering
MA 126 | Spring 2017 | Prof. Clontz | Readiness Quizzes
}}}
\end{center}
% \vspace{0.1in}
% \makebox[\textwidth]{
%   Name:\enspace\hrulefill\hrulefill\hrulefill
% }

\vspace{12pt}

\begin{itemize}
  \item Which of these is a definition of \(a^x\) for all positive numbers
        \(a\) and all real numbers \(x\)?
        \questionDate{2017-01-11 | practice}
  \begin{choices}
    \item \(\ln(x\cdot e^a)\)
    \item \(a\) multiplied by itself \(x\) times
    \item the unique function for which \(\frac{d}{dx}[a^x]=a^x\)
    \item \(\exp(x\ln a)\)
  \end{choices}
  \item Which of these statements is false?
        \questionDate{2017-01-11 | practice}
  \begin{choices}
    \item \(\ln(abc)=\ln(a)+\ln(b)+\ln(c)\)
    \item \(\frac{d}{dx}[\ln x] = \frac{1}{|x|}\) for all nonzero numbers \(x\)
    \item \(y=\exp(x)\) if and only if \(x=\ln(y)\)
    \item \(e^x=\exp(x)\)
  \end{choices}
\end{itemize}

% \begin{questions}
%
% \question
% Prove that \(\int_1^3\sqrt{4t^2-12t+10}\,dt\) is the arclength of the curve
% \(y=x^2-3x+4\) between \((1,2)\) and \((3,4)\).
%
% \question
% Use the polar arclength formula to show that the circumference of the
% circle \(r=4\sin\theta\) is \(4\pi\).
%
% \question
% Find \(\displaystyle\lim_{n\to\infty}\frac{\sin n + 3n^2}{n^2+1}\).
%
% \question
% Does \(\displaystyle\sum_{n=3}^\infty\left(\frac{6}{n}-\frac{6}{n+1}\right)\)
% converge or diverge? If it converges, what is its value?
%
% \question
% Does \(\displaystyle\sum_{n=1}^\infty\frac{1}{4^n}\) converge or
% diverge? If it converges, what is its value?
%
% \question
% Does \(\displaystyle\sum_{j=2}^\infty\frac{e^j}{e^{2j}+1}\) converge or diverge?
%
% \question
% Find the domain of
% \(\displaystyle f(x)=\sum_{m=2}^\infty\frac{(-2x)^m}{m}=
% \frac{4x^2}{2}-\frac{8x^3}{3}+\frac{16x^4}{4}-\frac{32x^5}{5}+\dots\).
%
% \question
% Generate the Maclaurin Series for \(\cosh x\).
%
% \question
% Use Taylor's Formula to
% approximate \(e^{-1/2}=\frac{1}{\sqrt e}\) with an error no greater than
% \(\frac{1}{1000}=0.001\).
%
% \end{questions}
%
% \vspace{3em}
%
% Reminder:
% These are examples of questions I may put on the final, so this is a good
% place to start studying, but this list
% is not exhaustive. You are responsible for understanding
% all material from Parts 4-6 of
% the course website, and for the types of problems which were asked on the
% Midterm.

\end{document}
