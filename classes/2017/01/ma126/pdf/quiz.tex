\documentclass[12pt]{exam}

\newcommand{\ds}{\ensuremath{\displaystyle}}

\newcommand{\questionDate}[1]{\textcolor{gray}{(#1)}}

\usepackage{amsmath,amsfonts, amsthm}
\usepackage{multicol}
\usepackage{multirow}
\usepackage{harpoon}
\renewcommand{\arraystretch}{1.5}

\newcommand{\harpvec}[1]{\overrightharp{\ensuremath{\mathbf{#1}}}}
\newcommand{\vect}[1]{\ensuremath{\mathbf{#1}}}
\newcommand{\<}{\langle}
\renewcommand{\>}{\rangle}
\newcommand{\p}{\partial}

% ref: http://pgfplots.sourceforge.net/gallery.html
% ref: http://tex.stackexchange.com/a/74575/79754
\usepackage{pgfplots}% This uses tikz
\pgfplotsset{compat=newest}% use newest version
\tikzset{LineStyle/.style={smooth, ultra thick, samples=400}}

% \printanswers

\begin{document}

\begin{center}
\fbox{\fbox{\parbox{5.5in}{\centering
MA 126 | Spring 2017 | Prof. Clontz | Readiness Quizzes
}}}
\end{center}
% \vspace{0.1in}
% \makebox[\textwidth]{
%   Name:\enspace\hrulefill\hrulefill\hrulefill
% }

\vspace{12pt}


\begin{itemize}
  \item Which of these is a definition of \(a^x\) for all positive numbers
        \(a\) and all real numbers \(x\)?
        \questionDate{2017-01-11, 1.1, practice}
  \begin{choices}
    \item \(\ln(x\cdot e^a)\)
    \item \(a\) multiplied by itself \(x\) times
    \item the unique function for which \(\frac{d}{dx}[a^x]=a^x\)
    \item \(\exp(x\ln a)\)
  \end{choices}
  \item Which of these statements is false?
        \questionDate{2017-01-11, 1.1, practice}
  \begin{choices}
    \item \(\ln(abc)=\ln(a)+\ln(b)+\ln(c)\)
    \item \(\frac{d}{dx}[\ln x] = \frac{1}{|x|}\) for all nonzero numbers \(x\)
    \item \(y=\exp(x)\) if and only if \(x=\ln(y)\)
    \item \(e^x=\exp(x)\)
  \end{choices}
\end{itemize}




\newpage

\begin{questions}

  \question
  The Substitution Rule is the opposite of which derivative rule?
  \questionDate{2017-01-19, 2.1}
  \begin{choices}
    \CorrectChoice Chain Rule
    \choice Product Rule
    \choice Quotient Rule
    \choice Power Rule
  \end{choices}

  \question
  What is incorrect about the following attempt at using the Substitution Rule?
  \[
    \int_0^1 (3-2x)^5\,dx = \int_0^1 u^5 \left(-\frac{1}{2}du\right)
  \]
  \questionDate{2017-01-19, 2.1}
  \begin{choices}
    \choice \(dx\) should have been replaced with \(+\frac{1}{2}\,du\).
    \choice \(u\) shouldn't be raised to the \(5\)th power.
    \choice \(dx\) should have been replaced with \(-2\,du\).
    \CorrectChoice The bounds are incorrect.
  \end{choices}





  \newpage

  \question
  Which of these formulas would be most useful in finding
  \(\int\sin^4\theta\cos^2\theta\,d\theta\)?
  \questionDate{2017-01-25, 2.2}
  \begin{choices}
    \choice \(\sin^2(\theta)=\frac{1}{2}+\frac{1}{2}\sin(2\theta)\)
    \CorrectChoice \(\cos^2(\theta)=\frac{1}{2}+\frac{1}{2}\cos(2\theta)\)
    \choice \(\cos^2(\theta)=1-\sin^2(\theta)\)
    \choice \(\sin^2(\theta)=1-\cos^2(\theta)\)
  \end{choices}

  \question
  Which of these formulas would be most useful in finding
  \(\int\sec^4(\theta)\,d\theta\)?
  \questionDate{2017-01-25, 2.2}
  \begin{choices}
    \CorrectChoice \(\sec^2(\theta)=1+\tan^2(\theta)\)
    \choice \(\sec^2(\theta)=1-\tan^2(\theta)\)
    \choice \(\tan^2(\theta)=1+\sec^2(\theta)\)
    \choice \(\tan^2(\theta)=1-\sec^2(\theta)\)
  \end{choices}







  \newpage

  \question
  Which of these substitutions would be most useful in finding
  \(\int\frac{1}{25x^2+9}\,dx\)?
  \questionDate{2017-01-27, 2.3}
  \begin{choices}
    \choice Let \(25x^2+9=25\sec^2\theta+25\).
    \choice Let \(25x^2+9=9\sin^2\theta+9\).
    \CorrectChoice Let \(25x^2+9=9\tan^2\theta+9\).
    \choice Let \(25x^2+9=25\cos^2\theta+25\).
  \end{choices}

  \question
  Which of these substitutions would be most useful in finding
  \(\int\frac{1}{x\sqrt{4-16x^2}}\,dx\)?
  \questionDate{2017-01-27, 2.3}
  \begin{choices}
    \choice Let \(4-16x^2=16-16\cos^2\theta\).
    \CorrectChoice Let \(4-16x^2=4-4\sin^2\theta\).
    \choice Let \(4-16x^2=4+4\tan^2\theta\).
    \choice Let \(4-16x^2=16+16\sec^2\theta\).
  \end{choices}

  \question
  Which of these substitutions would be most useful in finding
  \(\int_3^5\frac{1}{\sqrt{x^2-9}}\,dx\)?
  \questionDate{2017-01-27, 2.3}
  \begin{choices}
    \choice Let \(x^2-9=9\sin^2\theta+9\).
    \choice Let \(x^2-9=\tan^2\theta-1\).
    \choice Let \(x^2-9=\cos^2\theta+1\).
    \CorrectChoice Let \(x^2-9=9\sec^2\theta-9\).
  \end{choices}







  \newpage

  \question
  Which of these sums is the first step in expanding
  \(\frac{4x^2+16x+17}{(x+2)^2(x^2+1)^2}\) into partial fractions?
  \questionDate{2017-02-01, 2.4}
  \begin{choices}
    \choice \(\frac{A}{x+2}+\frac{Bx}{x+2}+\frac{C}{(x^2+1)^2}\)
    \choice \(\frac{A}{x+2}+\frac{Bx}{(x+2)^2}+\frac{Cx+D}{(x^2+1)^2}\)
    \CorrectChoice \(\frac{A}{x+2}+\frac{B}{(x+2)^2}+
      \frac{Cx+D}{x^2+1}+\frac{Ex+F}{(x^2+1)^2}\)
    \choice \(\frac{Ax+B}{(x+2)^2}+\frac{Cx+D}{(x^2+1)^2}\)
  \end{choices}

  \question
  Why must \(\frac{3+5x^5}{(x+1)(x+3)^2}\) first be simplifed using
  long polynomial division before using the method of partial fractions?
  \questionDate{2017-02-01, 2.4}
  \begin{choices}
    \choice It is a rational function of \(x\).
    \choice The degree of its numerator is odd, while the degree of its
      demoninator is even.
    \CorrectChoice The degree of its numerator is greater than or equal to
      the degree of its denominator.
    \choice It is an irrational function of \(x\).
  \end{choices}

\end{questions}


\end{document}
