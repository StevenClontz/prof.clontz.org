\documentclass[12pt]{article}

\usepackage[letterpaper,margin=1in]{geometry}

\setlength{\parindent}{0pt}

\usepackage{amssymb}
\usepackage{amsmath}

\usepackage{multicol}

\usepackage{tikz}

\newcommand{\headerText}{
  MA 126-103 | Summer 2017 | Dr. Clontz
}

\usepackage{fancyhdr}
\pagestyle{fancy}
\renewcommand{\headrulewidth}{0pt}% Default \headrulewidth is 0.4pt
\renewcommand{\footrulewidth}{0pt}% Default \footrulewidth is 0pt
\chead{\footnotesize\bf\headerText}
\cfoot{}

\newcommand{\csch}{\operatorname{csch}}
\newcommand{\sech}{\operatorname{sech}}


\newcommand{\exerciseHeader}[4]{

  \newpage

  \begin{tikzpicture}[x=1in,y=1in]
    \draw[color=black!20] (0,0) rectangle (4.5,1);
    \draw[color=black!20] (0,0.67) -- (4.5,0.66);
    \draw[color=black!20] (0,0.33) -- (4.5,0.33);

    \node[anchor=west] at (0.1,0.83) {Name:};
    \node[anchor=west] at (0.1,0.5) {J\#:};
    \node[anchor=west] at (0.1,0.17) {Date: \textbf{#1}};

    \node[anchor=west] at (4.6,0.83) {Exercise Type (Cost):};
    \node[anchor=west] at (4.7,0.5) {\textbf{In-Class (1AP)}};
  \end{tikzpicture}

  \vspace{1em}

  \begin{tikzpicture}[x=1in,y=1in]
    \draw[color=black!50] (0,0) rectangle (6.4,1);
    \draw[color=black!50] (5.4,0) -- (5.4,1);
    \draw[dashed,color=black!20] (5.4,0.25) -- (6.4,0.25);

    \node[anchor=north west,text width=4in,color=black!70] at (0,1) {\footnotesize Standard: This student is able to...};
    \node[anchor=north west,text width=4.5in] at (0.05,0.8) {\textbf{#2} #3};
    \node[anchor=south west,color=black!70] at (0, 0) {\footnotesize #4};

    \node[anchor=north west,color=black!70] at (5.4,0.95) {\footnotesize Mark:};
    \node[anchor=south east,color=black!70] at (5.4,0) {\footnotesize \(\star\) reattempt due on:};
  \end{tikzpicture}

  \vspace{1em}

}



\begin{document}

% \exerciseHeader{2017 June 01}{C01: LogExpDerInt.}{
% Find derivatives and integrals involving logrithmic and exponential functions.
% }{1/4}
%
% a) Find \(\frac{d}{dx}[ex+\ln(2x)]\).
%
% \vfill
%
% b) Find \(\int(\frac{4}{x}-3e^x)\,dx\).
%
% \vfill

% \exerciseHeader{2017 June 02}{C01: LogExpDerInt.}{
% Find derivatives and integrals involving logrithmic and exponential functions.
% }{2/4}
%
% a) Find \(\frac{d}{dt}[5e^{2t}+\frac{1}{t}]\).
%
% \vfill
%
% b) Find \(\int\frac{z^3+2z+4}{z}\,dz\).
%
% \vfill

% \exerciseHeader{2017 June 06}{C01: LogExpDerInt.}{
% Find derivatives and integrals involving logrithmic and exponential functions.
% }{3/4}
%
% Prove that \(\int (2ye^{y^2}+\frac{4}{y})\,dy=e^{y^2}+4\ln|3y|+C\).

% \exerciseHeader{2017 June 07}{C01: LogExpDerInt.}{
% Find derivatives and integrals involving logrithmic and exponential functions.
% }{4/4}
%
% Prove that \(\int \frac{x^2e^x+4x}{x^2}dx=e^x+\ln(x^4)+C\).



% \exerciseHeader{2017 June 02}{S01: LogExpPrf.}{
% Derive properties of the logarithmic and exponential functions from their definitions.
% }{1/3}
%
% Prove that \(\ln(ax)=\ln(a)+\ln(x)\).

% \exerciseHeader{2017 June 06}{S01: LogExpPrf.}{
% Derive properties of the logarithmic and exponential functions from their definitions.
% }{2/3}
%
% Use the definitions
% \( \log_b x = \frac{\ln x}{\ln b} \) and \(b^x = \exp(x\ln b)\)
% to prove the property \( x = \log_b(b^x) \). (That is, prove that
% \(\log_b x\) and \(b^x\) are inverse functions.)

% \exerciseHeader{2017 June 07}{S01: LogExpPrf.}{
% Derive properties of the logarithmic and exponential functions from their definitions.
% }{3/3}
%
% Use \(\frac{d}{dx}[\ln(x)]=\frac{1}{x}\) and \(\ln(1)=0\) to
% prove that \(\ln(\frac{x^2}{4})=2\ln(x)-\ln(4)\).



% \exerciseHeader{2017 June 06}{C02: HypDerInt.}{
% Find derivatives and integrals involving hypberbolic functions.
% }{1/4}
%
% a) Find \(\frac{d}{dx}[\cosh(3x^2+7)]\).
%
% \vfill
%
% b) Find \(\int(5\sech^2(x)-4\csch(x)\coth(x))\,dx\).
%
% \vfill

% \exerciseHeader{2017 June 07}{C02: HypDerInt.}{
% Find derivatives and integrals involving hypberbolic functions.
% }{2/4}
%
% a) Find \(\frac{d}{dx}[\sinh(2x)-\tanh(x)]\).
%
% \vfill
%
% b) Find \(\int(\cosh(x^2)\sech(x^2))\,dx\).
%
% \vfill

% \exerciseHeader{2017 June 08}{C02: HypDerInt.}{
% Find derivatives and integrals involving hypberbolic functions.
% }{3/4}
%
% a) Find \(\frac{d}{dx}[e^x\cosh(x)]\).
%
% \vfill
%
% b) Find \(\int(5\sinh(x)+\tanh(\pi))\,dx\).
%
% \vfill

% \exerciseHeader{2017 June 09}{C02: HypDerInt.}{
% Find derivatives and integrals involving hypberbolic functions.
% }{4/4}
%
% Show that \(\int4z\sech^2(z^2)\,dz=2\tanh(z^2)+C\).


% \exerciseHeader{2017 June 07}{S02: HypPrf.}{
% Prove hyperbolic function identities.
% }{1/3}
%
% Use the definitions
% \[
%   \sinh(x)=\frac{e^x-e^{-x}}{2}
%     \hspace{3em}
%   \cosh(x)=\frac{e^x+e^{-x}}{2}
% \]
% to prove that \(\sinh^2(x)+\cosh^2(x)=\cosh(2x)\).
%
% \end{document}

% \exerciseHeader{2017 June 08}{S02: HypPrf.}{
% Prove hyperbolic function identities.
% }{2/3}
%
% Use the identity
% \[
%   \cosh^2(x)-\sinh^2(x)=1
% \]
% to prove the identity \(\coth^2(x)=1+\csch^2(x)\).

% \exerciseHeader{2017 June 09}{S02: HypPrf.}{
% Prove hyperbolic function identities.
% }{3/3}
%
% Use the definitions
% \[
%   \sinh(x)=\frac{e^x-e^{-x}}{2}
%     \hspace{3em}
%   \cosh(x)=\frac{e^x+e^{-x}}{2}
% \]
% to prove that \(\frac{d}{dx}[\sinh(x)]=\cosh(x)\).



% \exerciseHeader{2017 June 08}{C03: IntSub.}{
% Use integration by substitution.
% }{1/4}
%
% Find \(\int 120z(2z+1)^4\,dz\).

% \exerciseHeader{2017 June 09}{C03: IntSub.}{
% Use integration by substitution.
% }{2/4}
%
% Find \(\int \frac{z^2+1}{z^3+3z+7}\,dz\).

% \exerciseHeader{2017 June 13}{C03: IntSub.}{
% Use integration by substitution.
% }{3/4}
%
% Find \(\int 6x^2\sin(x^3)\,dx\).

% \exerciseHeader{2017 June 14}{C03: IntSub.}{
% Use integration by substitution.
% }{4/4}
%
% Recall that \(\int \frac{1}{1+x^2}\,du=\tan^\leftarrow(x)+C\).
% Find \(\displaystyle\int \frac{2e^y}{e^{2y}+1}\,dy\).



% \exerciseHeader{2017 June 09}{S03: TrigId.}{
% Integrate products of trigonometric functions by applying trigonometric
% identities.
% }{1/3}
%
% Find \(\int\sin^3\theta\cos^2\theta\,d\theta\).

% \exerciseHeader{2017 June 13}{S03: TrigId.}{
% Integrate products of trigonometric functions by applying trigonometric
% identities.
% }{2/3}
%
% Find \(\int\cos^2\theta\,d\theta\).

% \exerciseHeader{2017 June 14}{S03: TrigId.}{
% Integrate products of trigonometric functions by applying trigonometric
% identities.
% }{3/3}
%
% Find \(\int\tan^3\theta\sec^2\theta\,d\theta\).



% \exerciseHeader{2017 June 13}{S04: TrigSub.}{
% Use trigonometric substitution.
% }{1/3}
%
% Find \(\displaystyle\int\frac{4}{9+4y^2}\,dy\).

% \exerciseHeader{2017 June 14}{S04: TrigSub.}{
% Use trigonometric substitution.
% }{2/3}
%
% Recall that \(\sin(2\theta)=2\sin\theta\cos\theta\)
% and \(\cos^2(\theta)=\frac{1}{2}+\frac{1}{2}\cos(2\theta)\).
% Find \(\displaystyle\int\frac{2x^2}{\sqrt{4-x^2}}\,dx\).
%
% \exerciseHeader{2017 June 15}{S04: TrigSub.}{
% Use trigonometric substitution.
% }{3/3}
%
% Find \(\displaystyle\int\frac{x}{4-x^2}\,dx\) using the trigonometric substitution
% \(4-x^2=4-4\sin^2\theta\) and the integral
% \(\int\tan\theta\,d\theta=-\ln|\cos\theta|+C\).



% \exerciseHeader{2017 June 14}{S05: PartFrac.}{
% Use partial fractions to integrate rational functions.
% }{1/3}
%
% Find \(\displaystyle\int\frac{2x+5}{x^2-x-2}\,dx\).

% \exerciseHeader{2017 June 15}{S05: PartFrac.}{
% Use partial fractions to integrate rational functions.
% }{2/3}
%
% Expand \(\displaystyle\frac{3x^2-2x+3}{(x^2+1)^2}\) using partial fractions.
% Do not integrate.

% \exerciseHeader{2017 June 16}{S05: PartFrac.}{
% Use partial fractions to integrate rational functions.
% }{3/3}
%
% Give the partial fraction expansion of
% \(\displaystyle\frac{f(x)}{x(x+1)^3(x^2+4)^2}\)
% in terms of the unknown constants \(A\) through \(H\), assuming
% \(f(x)\) is a polynomial of degree less than \(8\). Do not solve
% for \(A\) through \(H\).



% \exerciseHeader{2017 June 15}{C04: IntParts.}{
% Use integration by parts.
% }{1/4}
%
% Find \(\int 3x^2e^x\,dx\).

% \exerciseHeader{2017 June 16}{C04: IntParts.}{
% Use integration by parts.
% }{2/4}
%
% Find \(\int 8\sin(x)\cosh(x)\,dx\). (Note that one factor is trigonometric sine,
% the other is hyperbolic cosine, so integration by parts is in fact necessary.)

% \exerciseHeader{2017 June 19}{C04: IntParts.}{
% Use integration by parts.
% }{3/4}
%
% Find \(\int x\sin(2x)\,dx\).

% \exerciseHeader{2017 June 20}{C04: IntParts.}{
% Use integration by parts.
% }{4/4}
%
% Find \(\int 3x^2\ln(x)\,dx\).



% \exerciseHeader{2017 June 16}{C05: IntTech. }{
% Identify appropriate integration techniques.
% }{1/4}
%
% Draw lines matching each of the five integrals on the left with
% the most appropriate integration technique listed on the right.
% Multiple techniques may be technically possible, but choose the technique most
% useful to begin integration. Every integral and technique is used exactly
% once in the correct answer.
%
% \vspace{1em}
%
% \begin{multicols}{2}
%   \begin{itemize}
%     \item[] \(\displaystyle\int x\sin(x)\,dx\)
%     \item[] \(\displaystyle\int \cos^4(x)\,dx\)
%     \item[] \(\displaystyle\int\frac{1}{x\sqrt{x^2+1}}\,dx\)
%     \item[] \(\displaystyle\int\frac{4x}{3x^2-1}\,dx\)
%     \item[] \(\displaystyle\int\frac{7x^2+x+12}{x^3+3x}\,dx\)
%   \end{itemize}
%   \columnbreak
%   \begin{itemize}
%     \item Integration by Substiution
%     \item Method of Partial Fractions
%     \item Trigonometric Identities
%     \item Trigonometric Substitution
%     \item Integration by Parts
%   \end{itemize}
% \end{multicols}
%
% \exerciseHeader{2017 June 19}{C05: IntTech. }{
% Identify appropriate integration techniques.
% }{2/4}
%
% Draw lines matching each of the five integrals on the left with
% the most appropriate integration technique listed on the right.
% Multiple techniques may be technically possible, but choose the technique most
% useful to begin integration. Every integral and technique is used exactly
% once in the correct answer.
%
% \vspace{1em}
%
% \begin{multicols}{2}
%   \begin{itemize}
%     \item[] \(\displaystyle\int\frac{x^3}{\sqrt{4-x^2}}\,dx\)
%     \item[] \(\displaystyle\int x^2\cos(x)\,dx\)
%     \item[] \(\displaystyle\int\cos^2(x)\,dx\)
%     \item[] \(\displaystyle\int\frac{x+4}{x^2+3x+2}\,dx\)
%     \item[] \(\displaystyle\int x\cos(x^2)\,dx\)
%   \end{itemize}
%   \columnbreak
%   \begin{itemize}
%     \item Integration by Substiution
%     \item Method of Partial Fractions
%     \item Trigonometric Identities
%     \item Trigonometric Substitution
%     \item Integration by Parts
%   \end{itemize}
% \end{multicols}

% \exerciseHeader{2017 June 20}{C05: IntTech. }{
% Identify appropriate integration techniques.
% }{3/4}
%
% Draw lines matching each of the five integrals on the left with
% the most appropriate integration technique listed on the right.
% Multiple techniques may be technically possible, but choose the technique most
% useful to begin integration. Every integral and technique is used exactly
% once in the correct answer.
%
% \vspace{1em}
%
% \begin{multicols}{2}
%   \begin{itemize}
%     \item[] \(\displaystyle\int\frac{x^2-4x+7}{(x^2+4)(x-2)}\,dx\)
%     \item[] \(\displaystyle\int\frac{16x^2}{\sqrt{4+x^2}}\,dx\)
%     \item[] \(\displaystyle\int x^3e^x\,dx\)
%     \item[] \(\displaystyle\int \frac{\sin(\ln(x))}{x}\,dx\)
%     \item[] \(\displaystyle\int\frac{1}{\cos^4(x)}\,dx\)
%   \end{itemize}
%   \columnbreak
%   \begin{itemize}
%     \item Integration by Substiution
%     \item Method of Partial Fractions
%     \item Trigonometric Identities
%     \item Trigonometric Substitution
%     \item Integration by Parts
%   \end{itemize}
% \end{multicols}

\exerciseHeader{2017 June 21}{C05: IntTech. }{
Identify appropriate integration techniques.
}{4/4}

Draw lines matching each of the five integrals on the left with
the most appropriate integration technique listed on the right.
Multiple techniques may be technically possible, but choose the technique most
useful to begin integration. Every integral and technique is used exactly
once in the correct answer.

\vspace{1em}

\begin{multicols}{2}
  \begin{itemize}
    \item[] \(\displaystyle\int\sin^4(x)\cos^3(x)\,dx\)
    \item[] \(\displaystyle\int\frac{x^3+4x-1}{(x+4)(x^2+5)^2}\,dx\)
    \item[] \(\displaystyle\int 6x^2\sqrt{1+x^3}\,dx\)
    \item[] \(\displaystyle\int \sin(2x)e^x\,dx\)
    \item[] \(\displaystyle\int\frac{4}{x^2\sqrt{x^2-1}}\,dx\) where \(x>1\)
  \end{itemize}
  \columnbreak
  \begin{itemize}
    \item Integration by Substiution
    \item Method of Partial Fractions
    \item Trigonometric Identities
    \item Trigonometric Substitution
    \item Integration by Parts
  \end{itemize}
\end{multicols}



% \exerciseHeader{2017 June 19}{C06: AreaBtCurv. }{
% Express an area between curves as a definite integral.
% }{1/4}
%
% Find a definite integral equal to the area bounded by
% \(y=2x^2+x\) and \(y=x^2-x+3\).

% \exerciseHeader{2017 June 20}{C06: AreaBtCurv. }{
% Express an area between curves as a definite integral.
% }{2/4}
%
% Find a definite integral equal to the area bounded by
% \(y=x^3\) and \(y=x\).

\exerciseHeader{2017 June 21}{C06: AreaBtCurv. }{
Express an area between curves as a definite integral.
}{3/4}

Find a definite integral equal to the area bounded by
\(y=x\), \(y=2x-1\), and \(x=2\).



\exerciseHeader{2017 June 21}{S06: CrossSect.}{
Express an area between curves as a definite integral.
}{1/3}

Find a definite integral equal to the volume of the wedge-shaped
solid whose base lays on the region \(0\leq x\leq 4\) and \(0\leq y\leq 2\),
and whose cross-sections at each \(x\)-value are rectangles of height \(x\).


\end{document}
