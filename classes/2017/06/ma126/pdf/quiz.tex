\documentclass[12pt]{exam}

\newcommand{\ds}{\ensuremath{\displaystyle}}

\newcommand{\questionDate}[1]{\textcolor{gray}{(#1)}}

\usepackage{amsmath,amsfonts, amsthm}
\usepackage{multicol}
\usepackage{multirow}
\usepackage{harpoon}
\renewcommand{\arraystretch}{1.5}

\newcommand{\harpvec}[1]{\overrightharp{\ensuremath{\mathbf{#1}}}}
\newcommand{\vect}[1]{\ensuremath{\mathbf{#1}}}
\newcommand{\<}{\langle}
\renewcommand{\>}{\rangle}
\newcommand{\p}{\partial}

% ref: http://pgfplots.sourceforge.net/gallery.html
% ref: http://tex.stackexchange.com/a/74575/79754
\usepackage{pgfplots}% This uses tikz
\pgfplotsset{compat=newest}% use newest version
\tikzset{LineStyle/.style={smooth, ultra thick, samples=400}}

% \printanswers

\begin{document}

\begin{center}
\fbox{\fbox{\parbox{5.5in}{\centering
MA 126 | Spring 2017 | Prof. Clontz | Readiness Quizzes
}}}
\end{center}
% \vspace{0.1in}
% \makebox[\textwidth]{
%   Name:\enspace\hrulefill\hrulefill\hrulefill
% }

\vspace{12pt}


\begin{itemize}
  \item Which of these is a definition of \(a^x\) for all positive numbers
        \(a\) and all real numbers \(x\)?
        \questionDate{2017-01-11, 1.1, practice}
  \begin{choices}
    \item \(\ln(x\cdot e^a)\)
    \item \(a\) multiplied by itself \(x\) times
    \item the unique function for which \(\frac{d}{dx}[a^x]=a^x\)
    \item \(\exp(x\ln a)\)
  \end{choices}
  \item Which of these statements is false?
        \questionDate{2017-01-11, 1.1, practice}
  \begin{choices}
    \item \(\ln(abc)=\ln(a)+\ln(b)+\ln(c)\)
    \item \(\frac{d}{dx}[\ln x] = \frac{1}{|x|}\) for all nonzero numbers \(x\)
    \item \(y=\exp(x)\) if and only if \(x=\ln(y)\)
    \item \(e^x=\exp(x)\)
  \end{choices}
\end{itemize}




\newpage

\begin{questions}

  \question
  The Substitution Rule is the opposite of which derivative rule?
  \questionDate{2017-01-19, 2.1}
  \begin{choices}
    \CorrectChoice Chain Rule
    \choice Product Rule
    \choice Quotient Rule
    \choice Power Rule
  \end{choices}

  \question
  What is incorrect about the following attempt at using the Substitution Rule?
  \[
    \int_0^1 (3-2x)^5\,dx = \int_0^1 u^5 \left(-\frac{1}{2}du\right)
  \]
  \questionDate{2017-01-19, 2.1}
  \begin{choices}
    \choice \(dx\) should have been replaced with \(+\frac{1}{2}\,du\).
    \choice \(u\) shouldn't be raised to the \(5\)th power.
    \choice \(dx\) should have been replaced with \(-2\,du\).
    \CorrectChoice The bounds are incorrect.
  \end{choices}





  \newpage

  \question
  Which of these formulas would be most useful in finding
  \(\int\sin^4\theta\cos^2\theta\,d\theta\)?
  \questionDate{2017-01-25, 2.2}
  \begin{choices}
    \choice \(\sin^2(\theta)=\frac{1}{2}+\frac{1}{2}\sin(2\theta)\)
    \CorrectChoice \(\cos^2(\theta)=\frac{1}{2}+\frac{1}{2}\cos(2\theta)\)
    \choice \(\cos^2(\theta)=1-\sin^2(\theta)\)
    \choice \(\sin^2(\theta)=1-\cos^2(\theta)\)
  \end{choices}

  \question
  Which of these formulas would be most useful in finding
  \(\int\sec^4(\theta)\,d\theta\)?
  \questionDate{2017-01-25, 2.2}
  \begin{choices}
    \CorrectChoice \(\sec^2(\theta)=1+\tan^2(\theta)\)
    \choice \(\sec^2(\theta)=1-\tan^2(\theta)\)
    \choice \(\tan^2(\theta)=1+\sec^2(\theta)\)
    \choice \(\tan^2(\theta)=1-\sec^2(\theta)\)
  \end{choices}







  \newpage

  \question
  Which of these substitutions would be most useful in finding
  \(\int\frac{1}{25x^2+9}\,dx\)?
  \questionDate{2017-01-27, 2.3}
  \begin{choices}
    \choice Let \(25x^2+9=25\sec^2\theta+25\).
    \choice Let \(25x^2+9=9\sin^2\theta+9\).
    \CorrectChoice Let \(25x^2+9=9\tan^2\theta+9\).
    \choice Let \(25x^2+9=25\cos^2\theta+25\).
  \end{choices}

  \question
  Which of these substitutions would be most useful in finding
  \(\int\frac{1}{x\sqrt{4-16x^2}}\,dx\)?
  \questionDate{2017-01-27, 2.3}
  \begin{choices}
    \choice Let \(4-16x^2=16-16\cos^2\theta\).
    \CorrectChoice Let \(4-16x^2=4-4\sin^2\theta\).
    \choice Let \(4-16x^2=4+4\tan^2\theta\).
    \choice Let \(4-16x^2=16+16\sec^2\theta\).
  \end{choices}

  \question
  Which of these substitutions would be most useful in finding
  \(\int_3^5\frac{1}{\sqrt{x^2-9}}\,dx\)?
  \questionDate{2017-01-27, 2.3}
  \begin{choices}
    \choice Let \(x^2-9=9\sin^2\theta+9\).
    \choice Let \(x^2-9=\tan^2\theta-1\).
    \choice Let \(x^2-9=\cos^2\theta+1\).
    \CorrectChoice Let \(x^2-9=9\sec^2\theta-9\).
  \end{choices}







  \newpage

  \question
  Which of these sums is the first step in expanding
  \(\frac{4x^2+16x+17}{(x+2)^2(x^2+1)^2}\) into partial fractions?
  \questionDate{2017-02-01, 2.4}
  \begin{choices}
    \choice \(\frac{A}{x+2}+\frac{Bx}{x+2}+\frac{C}{(x^2+1)^2}\)
    \choice \(\frac{A}{x+2}+\frac{Bx}{(x+2)^2}+\frac{Cx+D}{(x^2+1)^2}\)
    \CorrectChoice \(\frac{A}{x+2}+\frac{B}{(x+2)^2}+
      \frac{Cx+D}{x^2+1}+\frac{Ex+F}{(x^2+1)^2}\)
    \choice \(\frac{Ax+B}{(x+2)^2}+\frac{Cx+D}{(x^2+1)^2}\)
  \end{choices}

  \question
  Why must \(\frac{3+5x^5}{(x+1)(x+3)^2}\) first be simplifed using
  long polynomial division before using the method of partial fractions?
  \questionDate{2017-02-01, 2.4}
  \begin{choices}
    \choice It is a rational function of \(x\).
    \choice The degree of its numerator is odd, while the degree of its
      demoninator is even.
    \CorrectChoice The degree of its numerator is greater than or equal to
      the degree of its denominator.
    \choice It is an irrational function of \(x\).
  \end{choices}







  \newpage

  \question
  Which of these techniques is most appropriate as the first step to find
  \(\int z^2\sin(z^3)\,dz\)?
  \questionDate{2017-02-09, 2.6}
  \begin{choices}
    \CorrectChoice
      Integration by Substitution
    \choice
      Integration by Parts
    \choice
      Method of Partial Fractions
    \choice
      Trigonometric Identities
  \end{choices}

  \question
  Which of these techniques is most appropriate as the first step to find
  \(\int \frac{t^2+3t+1}{t^3+t}\,dt\)?
  \questionDate{2017-02-09, 2.6}
  \begin{choices}
    \CorrectChoice
      Method of Partial Fractions
    \choice
      Trigonometric Substitution
    \choice
      Trigonometric Identities
    \choice
      Integration by Substitution
  \end{choices}







  \newpage

  \question
  Which of these integrals
  represents the area bounded by the curves \(x=y^2\) and \(x=4\)?
  \questionDate{2017-02-13, 3.1}
  \begin{choices}
    \CorrectChoice
      \(\int_0^4((\sqrt x)-(-\sqrt x))\,dx\)
    \choice
      \(\int_0^2((x^2)-(4))\,dx\)
    \choice
      \(\int_2^4((x^2)-(-x^2))\,dx\)
    \choice
      \(\int_2^0((x^2)-(\sqrt x))\,dx\)
  \end{choices}







  \newpage

  Answer the following questions about the solid of revolution obtained by
  rotating the triangle with vertices \((1,1),(2,2),(2,3)\) around
  the line \(y=-1\).

  \question
  Which of these curves should be used to find the outer radius \(R(x)\)?
  \questionDate{2017-02-16, 3.3}
  \begin{choices}
    \choice
      \(y=\frac{1}{2}x\)
    \CorrectChoice
      \(y=2x-1\)
    \choice
      \(y=-2x+2\)
    \choice
      \(y=-\frac{1}{2}x+1\)
  \end{choices}

  \question
  What formula should be used for \(R(x)\)?
  \questionDate{2017-02-16, 3.3}
  \begin{choices}
    \CorrectChoice
      \(R(x)=2x\)
    \choice
      \(R(x)=1-2x\)
    \choice
      \(R(x)=2-\frac{1}{2}x\)
    \choice
      \(R(x)=\frac{1}{2}x-1\)
  \end{choices}

  \question
   What are the correct bounds for the washer method integral?
  \questionDate{2017-02-16, 3.3}
  \begin{choices}
    \choice
      \(\pi\int_{-1}^3([R(x)]^2-[r(x)]^2)\,dx\)
    \choice
      \(\pi\int_2^3([R(x)]^2-[r(x)]^2)\,dx\)
    \CorrectChoice
      \(\pi\int_1^2([R(x)]^2-[r(x)]^2)\,dx\)
    \choice
      \(\pi\int_0^2([R(x)]^2-[r(x)]^2)\,dx\)
  \end{choices}

\newpage

  \question
  In the work integral \(\int_a^b F(x)\,dx\), the function \(F(x)\)
  represents...
  \questionDate{2017-02-24, 3.5}
  \begin{choices}
    \choice
      Friction
    \choice
      Speed
    \CorrectChoice
      Force
    \choice
      Mass
  \end{choices}

  \question
  When computing the work done in pumping water out of a container,
  the video suggests using which formula?
  \questionDate{2017-02-24, 3.5}
  \begin{choices}
    \CorrectChoice
      \(\int_{y=c}^{y=d}\,dW\)
    \choice
      \(\int_a^b F(x)\,dx\)
    \choice
      \(\int_a^b \frac{W(x)}{x}\,dx\)
    \choice
      \(\int_{y=c}^{y=d}yF(y)\,dy\)
  \end{choices}

\newpage

  \question
  The equations \(x=3+4\cos(t),y=-2+4\sin(t)\) for \(0\leq t\leq 2\pi\)
  parametrize which kind of curve?
  \questionDate{2017-03-01, 4.1}
  \begin{choices}
    \choice
      A parabola
    \choice
      A line segment
    \CorrectChoice
      A circle oriented counter-clockwise
    \choice
      A circle oriented clockwise
  \end{choices}

  \question
  The equations \(x=3+4t,y=-2+4t\) for \(0\leq t\leq 2\pi\)
  parametrize which kind of curve?
  \questionDate{2017-03-01, 4.1}
  \begin{choices}
    \choice
      A parabola
    \choice
      A line segment
    \CorrectChoice
      A circle oriented counter-clockwise
    \choice
      A circle oriented clockwise
  \end{choices}

\newpage

  \question
  The formula \(\frac{dy}{dt}=\frac{dy}{dx}\frac{dx}{dt}\) for computing the
  slope of a tangent line to a curve defined by parametric equations is
  a result of the...
  \questionDate{2017-03-02, 4.2}
  \begin{choices}
    \choice
      Product Rule
    \choice
      Pythagorean Theorem
    \choice
      Method of Partial Fractions
    \CorrectChoice
      Chain Rule
  \end{choices}

  \question
  The formula \(L=\int_a^b\sqrt{[dx/dt]^2+[dy/dt]^2}\,dt\) for computing
  arclength is a result of the...
  \questionDate{2017-03-02, 4.2}
  \begin{choices}
    \choice
      Product Rule
    \CorrectChoice
      Pythagorean Theorem
    \choice
      Method of Partial Fractions
    \choice
      Chain Rule
  \end{choices}

\newpage

  \question
  Find the first few terms of the sequence defined recursively by
  \(a_0=1\), \(a_1=2\), \(a_{n+2}=2a_n+a_{n+1}\).
  \questionDate{2017-03-22, 5.1}
  \begin{choices}
    \choice
      \(\<1,2,3,4,5,\dots\>\)
    \choice
      \(\<1,2,3,5,8,\dots\>\)
    \CorrectChoice
      \(\<1,2,4,8,16,\dots\>\)
    \choice
      \(\<1,2,4,7,15,\dots\>\)
  \end{choices}

  \question
  The limit \(\displaystyle\lim_{n\to\infty}\frac{n}{1+n^2}\) is equal to
  which of the following limits?
  \questionDate{2017-03-22, 5.2}
  \begin{choices}
    \choice
      \(\displaystyle\lim_{n\to 0}\frac{1+n^2}{n}\)
    \CorrectChoice
      \(\displaystyle\lim_{x\to\infty}\frac{x}{1+x^2}\)
    \choice
      \(\displaystyle\lim_{x\to 0} \left(x+\frac{1}{x}\right)\)
    \choice
      \(\displaystyle\lim_{n\to\infty}\left(n+\frac{1}{n}\right)\)
  \end{choices}

  \question
  Which of the following describes the sequence
  \(
    \<(-\frac{2}{3})^n\>_{n=0}^\infty
  =
    \<1,-\frac{2}{3},\frac{4}{9},-\frac{8}{27},\dots\>
  \)?
  \questionDate{2017-03-22, 5.2}
  \begin{choices}
    \choice
      It is bounded and monotonic, and therefore convergent
      by the Monotonic Sequence Theorem.
    \CorrectChoice
      It is bounded and convergent, but not monotonic.
    \choice
      It is monotonic, but not bounded nor convergent.
    \choice
      It is convergent and monotonic, but not bounded.
  \end{choices}


\newpage

\question
Which of the following statements about the sequence
\(\<(\frac{1}{2})^n\>_{n=0}^\infty =
\<1,\frac{1}{2},\frac{1}{4},\frac{1}{8},\dots\>\) is false?
\questionDate{2017-03-27, 5.3}
\begin{choices}
  \choice
    The sequence is bounded and monotonic, and therefore convergent
    by the Monotonic Sequence Theorem.
  \choice
    Its partial sum sequence
    \(\<1,1+\frac{1}{2},1+\frac{1}{2}+\frac{1}{4},\dots\>=
    \<1,\frac{3}{2},\frac{7}{4},\dots\>\) is bounded and monotonic, and
    therefore convergent by the Monotonic Sequence Theorem.
  \choice
    Its corresponding series
    \(\sum_{n=0}^\infty (\frac{1}{2})^n=1+\frac{1}{2}+\frac{1}{4}+\dots\)
    converges to \(\frac{1}{1-\frac{1}{2}}=2\).
  \CorrectChoice
    Its corresponding series
    \(\sum_{n=0}^\infty (\frac{1}{2})^n=1+\frac{1}{2}+\frac{1}{4}+\dots\)
    is an infinite sum and therefore does not exist.
\end{choices}


\end{questions}


\end{document}
