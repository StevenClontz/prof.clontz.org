\documentclass[12pt]{article}

\usepackage[letterpaper,margin=1in]{geometry}

\setlength{\parindent}{0pt}

\usepackage{amssymb}
\usepackage{amsmath}

\usepackage{multicol}

\newcommand{\assessmentTitle}{
  Standard Assessment 4
}

\usepackage{fancyhdr}
\pagestyle{fancy}
\renewcommand{\headrulewidth}{0pt}% Default \headrulewidth is 0.4pt
\renewcommand{\footrulewidth}{0pt}% Default \footrulewidth is 0pt
\chead{\footnotesize\bf\assessmentTitle}
\cfoot{\small Page \thepage}

\newcommand{\makeHeader}[4]{
\thispagestyle{empty}
\begin{center}
\fbox{\fbox{\parbox{5.5in}{\centering
#1 | #2 | #3 | #4
}}}
\end{center}
\vspace{0.1in}
\makebox[\textwidth]{
  Name:\enspace\hrulefill\hrulefill\hrulefill
}
}

\usepackage{xcolor}

\newcommand{\standardQuestion}[2]{
\newpage
\begin{center}
  \begin{tabular}{|l|c|c|}
  \hline
    \parbox{4in}{
      \textbf{#1}: This student is able to...\\
      #2
    }
  &
    \parbox{1in}{
      \vspace{0.1in}
      \footnotesize \textcolor{gray}{Mark:}
      \vspace{0.7in}

      \tiny \textcolor{gray}{(Instructor Use Only)}
    }
  &
    \parbox{1in}{
      \vspace{0.1in}
      \footnotesize \textcolor{gray}{Reattempt/ Correction:}
      \vspace{0.53in}

      \tiny \textcolor{gray}{(Instructor Use Only)}
    }
  \\\hline
  \end{tabular}
\end{center}
}

\newcommand{\csch}{\operatorname{csch}}
\newcommand{\sech}{\operatorname{sech}}



\begin{document}

\makeHeader{
MA 126 }{ Spring 2017 }{ Prof. Clontz }{ \assessmentTitle
}

\begin{itemize}
  \item Each question is prefaced with a Standard for this course.
  \item When grading, each response will be marked as follows:
  \begin{itemize}
    \item \(\checkmark\):
      The response is demonstrates complete understanding of the Standard.
    \item \(\star\):
      The response may indicate full understanding of the Standard, but
      clarification or minor corrections are required.
    \item \(\times\):
      The response does not demonstrate complete understanding of the Standard.
  \end{itemize}
  \item Only responses marked with a \(\checkmark\) mark count toward your
    grade for the semester.
    Visit the course website for more information on how to improve
    \(\star\) and \(\times\) marks.
  \item This Assessment is due after 50 minutes. All blank responses will
    be marked with \(\times\).
\end{itemize}





% \standardQuestion{C01}{
%   Derive properties of the logarithmic and exponential functions from their definitions.
% }
%
% Let \(f^\leftarrow\) denote the inverse function of an invertable function
% \(f\); in particular, if \(f(x)=\ln(x)\), then \(f^\leftarrow(x)=\exp(x)\).
%
% Use the theorem
% \(\frac{d}{dx}[f^\leftarrow(x)]=\frac{1}{f'(f^\leftarrow(x))}\) to prove that
% \(\frac{d}{dx}[\exp x]=\exp x\).
%
%
%
%
%
% \standardQuestion{C02}{
%   Prove hyperbolic function identities.
% }
%
% Use the definitions
% \[
%   \tanh(x) = \frac{e^x-e^{-x}}{e^x+e^{-x}},
%   \sech(x) = \frac{2}{e^x+e^{-x}}
% \]
% to prove the following identity.
% \[
%   1-\tanh^2(x)=\sech^2(x)
% \]



\standardQuestion{C03}{
  Use integration by substitution.
}

Find \(\displaystyle\int \frac{6x^2+14}{x^3+7x-3}\,dx\).



\standardQuestion{C04}{
  Use integration by parts.
}

Find \(\int 8e^y\sin(y)\,dy\).



\standardQuestion{C05}{
  Identify and use appropriate integration techniques.
}

Draw lines matching each of the five integrals on the left with
the most appropriate integration technique listed on the right.
Multiple techniques may be technically possible, but choose the technique most
useful to begin integration. Every integral and technique is used exactly
once in the correct answer.

\vspace{1em}

\begin{multicols}{2}
  \begin{itemize}
    \item[] \(\displaystyle\int \sin^3(x)\cos^4(x)\,dx\)
    \item[] \(\displaystyle\int \sqrt{4-x^2}\,dx\)
    \item[] \(\displaystyle\int x\ln(x)\,dx\)
    \item[] \(\displaystyle\int 5x(3x^2-8)^9\,dx\)
    \item[] \(\displaystyle\int \frac{2x+7}{(x^2+1)(x-1)}\,dx\)
  \end{itemize}
  \columnbreak
  \begin{itemize}
    \item Integration by Substiution
    \item Method of Partial Fractions
    \item Trigonometric Identities
    \item Trigonometric Substitution
    \item Integration by Parts
  \end{itemize}
\end{multicols}



\standardQuestion{C06}{
  Express an area between curves as a definite integral.
}

Find a definite integral equal to the area between the curves
\(y=2\), \(x=0\), and \(x=y^2\). (Do not solve your integral.)



\standardQuestion{C07}{
  Use the washer or cylindrical shell method to express a volume of
  revolution as a definite integral.
}

Find a definite integral equal to the volume of
the solid of revolution obtained by rotating
the triangle with vertices \((1,1)\), \((2,0)\), and \((3,1)\)
around the axis \(x=1\). (Do not solve your integral.)




\standardQuestion{C08}{
  Express the work done in a system as a definite integral.
}

Find a definite integral equal to the work (in kilojoules) required to
pump the contents of a cylindrical tank with a radius of \(2\)
meters and a total height of \(8\) meters to its top, assuming it is initially
half-full of salt water that weighs \(10\) kilonetwons per square meter.
(Do not solve your integral.)





% \standardQuestion{S01}{
%   Find derivatives and integrals involving logrithmic and exponential functions.
% }
%
% a) Find \(\frac{d}{dz}[\ln(3e^z)]\).
%
% \vfill
%
% b) Find \(\displaystyle\int\left(2e+\frac{3}{y}\right)\,dy\).
%
% \vfill
%
%
%
% \standardQuestion{S02}{
%   Find derivatives and integrals involving hypberbolic functions.
% }
%
% a) Find \(\frac{d}{dv}[4\tanh(3v)-\sinh(v^2)]\).
%
% \vfill
%
% b) Find \(\displaystyle\int(\cosh(x)+2\sinh(x))\,dx\).
%
% \vfill



% \standardQuestion{S03}{
%   Integrate products of trigonometric functions.
% }
%
% Find \(\int2\cos^2(y)\,dy\).
%
%
%
% \standardQuestion{S04}{
%   Use trigonometric substitution.
% }
%
% Find \(\displaystyle\int\frac{z+1}{\sqrt{1-z^2}}\,dz\).
%
%
%
%
% \standardQuestion{S05}{
%   Use partial fractions to integrate rational functions.
% }
%
% a) Complete the following partial fraction expansion:
%
% \[
%   \frac{f(x)}{(x^2+3)^2(x-7)^4}
%     =
%   \frac{\hspace{3em}}{x^2+3}
%     +
%   \frac{\hspace{3em}}{(x^2+3)^2}
%     +
%   \frac{\hspace{3em}}{x-7}
%     +
%   \frac{\hspace{3em}}{(x-7)^2}
%     +
%   \frac{\hspace{3em}}{(x-7)^3}
%     +
%   \frac{\hspace{3em}}{(x-7)^4}
% \]
%
% \vspace{1em}
%
% (Assume the degree of \(f\) is less than \(8\).
% You do NOT need to solve for your constants.)
%
% \vspace{1em}
%
% b) Find \(\displaystyle\int\frac{3x^2-x+2}{(x^2+1)(x-1)}\,dx\).
%
% \vfill




\standardQuestion{S06}{
  Use cross-sectioning to express a volume as a definite integral.
}

Find a definite integral that equals the volume of a solid whose base
is the triangle with vertices \((0,0)\), \((1,2)\), and \((1,-2)\),
and whose cross-sections perpindicular to the \(x\)-axis are semicircles
with diameters on the \(xy\) plane. (Do not solve your integral.)




\standardQuestion{S07}{
  Derive a formula for the volume of a three dimensional solid.
}

Prove that the volume of a hemisphere (half-sphere) with radius
\(R\) is given by \(V=\frac{2}{3}\pi R^3\).





\standardQuestion{S08}{
  Parametrize planar curves and sketch parametrized curves.
}

a) Give a parameterization of the parabola \(y=x^2\) from \((-1,1)\)
   to \((3,9)\).

\vfill

b) Sketch the curve parameterized by \(x=2\sin\theta\), \(y=2\cos\theta\) for
   \(0\leq\theta\leq\pi\).

\vfill







\newpage

\textcolor{gray}{Use this space if you need extra room for a problem:}

\end{document}
