\documentclass[12pt]{article}

\usepackage[letterpaper,margin=1in]{geometry}

\setlength{\parindent}{0pt}

\usepackage{amssymb}
\usepackage{amsmath}

\usepackage{multicol}

\usepackage{tikz}

\newcommand{\headerText}{
  MA 126-103 | Summer 2017 | Dr. Clontz
}

\usepackage{fancyhdr}
\pagestyle{fancy}
\renewcommand{\headrulewidth}{0pt}% Default \headrulewidth is 0.4pt
\renewcommand{\footrulewidth}{0pt}% Default \footrulewidth is 0pt
\chead{\footnotesize\bf\headerText}
\cfoot{}

\newcommand{\csch}{\operatorname{csch}}
\newcommand{\sech}{\operatorname{sech}}


\newcommand{\exerciseHeader}[4]{

  \newpage

  \begin{tikzpicture}[x=1in,y=1in]
    \draw[color=black!20] (0,0) rectangle (4.5,1);
    \draw[color=black!20] (0,0.67) -- (4.5,0.66);
    \draw[color=black!20] (0,0.33) -- (4.5,0.33);

    \node[anchor=west] at (0.1,0.83) {Name:};
    \node[anchor=west] at (0.1,0.5) {J\#:};
    \node[anchor=west] at (0.1,0.17) {Date: \textbf{#1}};

    \node[anchor=west] at (4.6,0.83) {Exercise Type (Cost):};
    \node[anchor=west] at (4.7,0.5) {\textbf{In-Class (1AP)}};
  \end{tikzpicture}

  \vspace{1em}

  \begin{tikzpicture}[x=1in,y=1in]
    \draw[color=black!50] (0,0) rectangle (6.4,1);
    \draw[color=black!50] (5.4,0) -- (5.4,1);
    \draw[dashed,color=black!20] (5.4,0.25) -- (6.4,0.25);

    \node[anchor=north west,text width=4in,color=black!70] at (0,1) {\footnotesize Standard: This student is able to...};
    \node[anchor=north west,text width=4.5in] at (0.05,0.8) {\textbf{#2} #3};
    \node[anchor=south west,color=black!70] at (0, 0) {\footnotesize #4};

    \node[anchor=north west,color=black!70] at (5.4,0.95) {\footnotesize Mark:};
    \node[anchor=south east,color=black!70] at (5.4,0) {\footnotesize \(\star\) reattempt due on:};
  \end{tikzpicture}

  \vspace{1em}

}



\begin{document}

% \exerciseHeader{2017 June 01}{C01: LogExpDerInt.}{
% Find derivatives and integrals involving logrithmic and exponential functions.
% }{1/4}
%
% a) Find \(\frac{d}{dx}[ex+\ln(2x)]\).
%
% \vfill
%
% b) Find \(\int(\frac{4}{x}-3e^x)\,dx\).
%
% \vfill

% \exerciseHeader{2017 June 02}{C01: LogExpDerInt.}{
% Find derivatives and integrals involving logrithmic and exponential functions.
% }{2/4}
%
% a) Find \(\frac{d}{dt}[5e^{2t}+\frac{1}{t}]\).
%
% \vfill
%
% b) Find \(\int\frac{z^3+2z+4}{z}\,dz\).
%
% \vfill

% \exerciseHeader{2017 June 06}{C01: LogExpDerInt.}{
% Find derivatives and integrals involving logrithmic and exponential functions.
% }{3/4}
%
% Prove that \(\int (2ye^{y^2}+\frac{4}{y})\,dy=e^{y^2}+4\ln|3y|+C\).

\exerciseHeader{2017 June 07}{C01: LogExpDerInt.}{
Find derivatives and integrals involving logrithmic and exponential functions.
}{4/4}

Prove that \(\int \frac{x^2e^x+4x}{x^2}dx=e^x+\ln(x^4)+C\).



% \exerciseHeader{2017 June 02}{S01: LogExpPrf.}{
% Derive properties of the logarithmic and exponential functions from their definitions.
% }{1/3}
%
% Prove that \(\ln(ax)=\ln(a)+\ln(x)\).

% \exerciseHeader{2017 June 06}{S01: LogExpPrf.}{
% Derive properties of the logarithmic and exponential functions from their definitions.
% }{2/3}
%
% Use the definitions
% \( \log_b x = \frac{\ln x}{\ln b} \) and \(b^x = \exp(x\ln b)\)
% to prove the property \( x = \log_b(b^x) \). (That is, prove that
% \(\log_b x\) and \(b^x\) are inverse functions.)

\exerciseHeader{2017 June 07}{S01: LogExpPrf.}{
Derive properties of the logarithmic and exponential functions from their definitions.
}{3/3}

Use \(\frac{d}{dx}[\ln(x)]=\frac{1}{x}\) and \(\ln(1)=0\) to
prove that \(\ln(\frac{x^2}{4})=2\ln(x)-\ln(4)\).



% \exerciseHeader{2017 June 06}{C02: HypDerInt.}{
% Find derivatives and integrals involving hypberbolic functions.
% }{1/4}
%
% a) Find \(\frac{d}{dx}[\cosh(3x^2+7)]\).
%
% \vfill
%
% b) Find \(\int(5\sech^2(x)-4\csch(x)\coth(x))\,dx\).
%
% \vfill

\exerciseHeader{2017 June 06}{C02: HypDerInt.}{
Find derivatives and integrals involving hypberbolic functions.
}{2/4}

a) Find \(\frac{d}{dx}[\sinh(2x)-\tanh(x)]\).

\vfill

b) Find \(\int(\cosh(x^2)\sech(x^2))\,dx\).

\vfill



\exerciseHeader{2017 June 06}{S02: HypPrf.}{
Prove hyperbolic function identities.
}{1/3}

Use the definitions
\[
  \sinh(x)=\frac{e^x-e^{-x}}{2}
    \hspace{3em}
  \cosh(x)=\frac{e^x+e^{-x}}{2}
\]
to prove that \(\sinh^2(x)+\cosh^2(x)=\cosh(2x)\).

\end{document}
