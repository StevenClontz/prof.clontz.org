\documentclass[12pt]{article}

\usepackage[letterpaper,margin=1in]{geometry}

\setlength{\parindent}{0pt}

\usepackage{amssymb}
\usepackage{amsmath}

\usepackage{multicol}

\usepackage{tikz}

\newcommand{\headerText}{
  MA 227-103 | Summer 2017 | Dr. Clontz
}

\usepackage{fancyhdr}
\pagestyle{fancy}
\renewcommand{\headrulewidth}{0pt}% Default \headrulewidth is 0.4pt
\renewcommand{\footrulewidth}{0pt}% Default \footrulewidth is 0pt
\chead{\footnotesize\bf\headerText}
\cfoot{}

\newcommand{\csch}{\operatorname{csch}}
\newcommand{\sech}{\operatorname{sech}}

\newcommand{\vect}{\mathbf}
\newcommand{\<}{\left\langle}
\renewcommand{\>}{\right\rangle}
\newcommand{\veci}{\hat{\imath}}
\newcommand{\vecj}{\hat{\jmath}}
\newcommand{\veck}{\hat{k}}

\newcommand{\exerciseHeader}[4]{

  \newpage

  \begin{tikzpicture}[x=1in,y=1in]
    \draw[color=black!20] (0,0) rectangle (4.5,1);
    \draw[color=black!20] (0,0.67) -- (4.5,0.66);
    \draw[color=black!20] (0,0.33) -- (4.5,0.33);

    \node[anchor=west] at (0.1,0.83) {Name:};k
    \node[anchor=west] at (0.1,0.5) {J\#:};
    \node[anchor=west] at (0.1,0.17) {Date: \textbf{#1}};

    \node[anchor=west] at (4.6,0.83) {Exercise Type:};
    \node[anchor=west] at (4.7,0.5) {\textbf{Quiz}};
  \end{tikzpicture}

  \vspace{1em}

  \begin{tikzpicture}[x=1in,y=1in]
    \draw[color=black!50] (0,0) rectangle (6.4,1);
    \draw[color=black!50] (5.4,0) -- (5.4,1);
    \draw[dashed,color=black!20] (5.4,0.25) -- (6.4,0.25);

    \node[anchor=north west,text width=4in,color=black!70] at (0,1) {\footnotesize Standard: This student is able to...};
    \node[anchor=north west,text width=4.5in] at (0.05,0.8) {\textbf{#2} #3};
    \node[anchor=south west,color=black!70] at (0, 0) {\footnotesize #4};

    \node[anchor=north west,color=black!70] at (5.4,0.95) {\footnotesize Mark:};
    \node[anchor=south east,color=black!70] at (5.4,0) {\footnotesize \(\star\) reattempt due on:};
  \end{tikzpicture}

  \vspace{1em}

}



\begin{document}

% \exerciseHeader{2017 June 01}{S01: 3DSpace.}{
% Plot and analyze points and vectors in three-dimensional Euclidean space.
% }{1/3}
%
% Find the magnitude \(\|\vect v\|\) and
% direction \(\frac{1}{\|\vect v\|}\vect v\)
% of the vector \(\vect v = \<2,-4,4\>\).

% \exerciseHeader{2017 June 02}{S01: 3DSpace.}{
% Plot and analyze points and vectors in three-dimensional Euclidean space.
% }{2/3}
%
% Sketch the vector \(\vect v=\<0,2,-3\>\) in \(xyz\) space.
% Then compute \(3\vect v\), and sketch it in \(xyz\) space
% as well.

% \exerciseHeader{2017 June 05}{S01: 3DSpace.}{
% Plot and analyze points and vectors in three-dimensional Euclidean space.
% }{3/3}
%
% In \(xyz\) space,
% sketch the vector \(\vect v=\<1,3,0\>\), the vector \(\vect w\) pointing
% from \(\<1,3,0\>\) to \(\<1,3,4\>\), and the vector \(\vect v+\vect w\).



% \exerciseHeader{2017 June 02}{S02: DotProd.}{
% Compute and apply the dot product of two vectors.
% }{1/3}
%
% Find \(\cos\theta\), where \(\theta\) is the angle between the vectors
% \(\<5,0,-12\>\) and \(\<3,-4,0\>\).

% \exerciseHeader{2017 June 05}{S02: DotProd.}{
% Compute and apply the dot product of two vectors.
% }{2/3}
%
% Find the work done by a force of \(6\) units over a distance of \(4\) units,
% assuming that the force vector is applied at an angle of \(\pi/3\) radians from
% the displacement vector.

% \exerciseHeader{2017 June 06}{S02: DotProd.}{
% Compute and apply the dot product of two vectors.
% }{3/3}
%
% Verify that
% \[
%   \<3,0,-2\>\cdot(\<-1,2,3\>+\<3,1,4\>)
% =
%   \<3,0,-2\>\cdot\<-1,2,3\>+\<3,0,-2\>\cdot\<3,1,4\>
% \]
% by computing both sides separately.



% \exerciseHeader{2017 June 05}{S03: CrossProd.}{
% Compute and apply the cross product of two vectors.
% }{1/3}
%
% Prove that \(\veci\times\veck=-\vecj\), either by computing the cross product
% directly, or by using
% \(\vect{v}\times\vect{w}=(\|\vect v\|\|\vect w\|\sin\theta)\vect{n}\).

% \exerciseHeader{2017 June 06}{S03: CrossProd.}{
% Compute and apply the cross product of two vectors.
% }{2/3}
%
% Prove that \(\<2,-6,4\>\) and \(\<-3,9,-6\>\) are parallel vectors.

% \exerciseHeader{2017 June 07}{S03: CrossProd.}{
% Compute and apply the cross product of two vectors.
% }{3/3}
%
% A force of \(8\) units is applied to a wrench at an angle of \(\pi/4\)
% radians to a point \(3\sqrt 2\) units away from a bolt.
% What is the mangitude of the resulting torque?



% \exerciseHeader{2017 June 07}{C01: SurfaceEQ.}{
% Identify and sketch surfaces in three-dimensional Euclidean space.
% }{1/4}
%
% Sketch the surface \((x-4)^2+y^2+(z+3)^2=4\) in \(xyz\) space.

% \exerciseHeader{2017 June 08}{C01: SurfaceEQ.}{
% Identify and sketch surfaces in three-dimensional Euclidean space.
% }{2/4}
%
% Consider the quadric surface \(z=y^2-x^2\).
%
% \vspace{1em}
%
% First sketch six traces for the surface given by \(x=-2,x=0,x=2\) and
% \(y=-2,y=0,y=2\) in two dimensions.
%
% \vfill\vfill
%
% Then use those traces to draw a rough three-dimensional sketch of the surface.
% (This quadric surface is called a hyperbolic paraboloid.)
%
% \vfill

% \exerciseHeader{2017 June 09}{C01: SurfaceEQ.}{
% Identify and sketch surfaces in three-dimensional Euclidean space.
% }{3/4}
%
% Sketch the surface \((x-2)+(y-2)+(z-2)=0\), labeling both a point on
% the surface and a normal vector to the surface at that point.

% \exerciseHeader{2017 June 12}{C01: SurfaceEQ.}{
% Identify and sketch surfaces in three-dimensional Euclidean space.
% }{4/4}
%
% Sketch the equation \(x^2+z^2=4\) first as a curve in the \(xz\) plane,
% then as a surface in \(xyz\) space.



% \exerciseHeader{2017 June 08}{C02: VectFunc.}{
% Model curves in Euclidean space with vector functions.
% }{1/4}
%
% Give a vector function modeling the line passing through
% \(\<1,2,3\>\) and parallel to the line with vector function
% \(\vect r(t)=\<3-t,5+2t,4t\>\).

% \exerciseHeader{2017 June 09}{C02: VectFunc.}{
% Model curves in Euclidean space with vector functions.
% }{2/4}
%
% Give a vector function parameterizing the portion of the parabola \(y=x^2+4\)
% beginning at \(\<-1,5\>\) and ending at \(\<2,8\>\).

% \exerciseHeader{2017 June 12}{C02: VectFunc.}{
% Model curves in Euclidean space with vector functions.
% }{3/4}
%
% Give a vector function parameterizing the line segment beginning at
% \(\<1,2,3\>\) and ending at \(\<4,0,-2\>\).

% \exerciseHeader{2017 June 13}{C02: VectFunc.}{
% Model curves in Euclidean space with vector functions.
% }{4/4}
%
% Give a vector function parameterizing the circle with center
% \(\<1,2\>\) and passing through the point \(\<-2,6\>\).



% \exerciseHeader{2017 June 09}{C03: VectCalc.}{
% Compute and apply vector function limits, derivatives, and integrals.
% }{1/4}
%
% Find the limit of \(\vect r(t)=\<\frac{\sin(t-1)}{t^2},\frac{3t^2-3t}{t^2-1}\>\)
% as \(t\) approaches \(1\).

% \exerciseHeader{2017 June 12}{C03: VectCalc.}{
% Compute and apply vector function limits, derivatives, and integrals.
% }{2/4}
%
% Find a vector tangent to the curve parameterized by
% \(\vect r(t)=\<t^2,t^3,t\>\) at the point \(\<4,8,2\>\).

% \exerciseHeader{2017 June 13}{C03: VectCalc.}{
% Compute and apply vector function limits, derivatives, and integrals.
% }{3/4}
%
% Find \(\int\vect r(t)\,dt\) where \(\vect r(t)=\<\cos(t),6t^2,e^t\>\).

% \exerciseHeader{2017 June 14}{C03: VectCalc.}{
% Compute and apply vector function limits, derivatives, and integrals.
% }{4/4}
%
% Find \(\vect r(t)\) given \(\vect r'(t)=\<-\sin(t),4+2t\>\) and
% \(\vect r(0)=\<3,2\>\).



% \exerciseHeader{2017 June 13}{S04: Kinematics.}{
% Compute and apply position, velocity, and acceleration vector functions.
% }{1/3}
%
% Recall that position in ideal projectile motion is given by
% \(\vect r(t) = P_0+\vect{v}_0 t-\frac{1}{2}g\vecj t^2\) where
% \(P_0\) is the initial position, \(\vect{v}_0\) is initial velocity,
% and \(g\) is acceleration due to gravity.
%
% \vspace{1em}
%
% Assume \(g=10\) meters per second squared.
% Find the speed of a projectile after \(3\) seconds if it is launched
% from the ground
% with initial speed \(20\sqrt{3}\)
% meters per second at an angle of \(\pi/3\) radians.

% \exerciseHeader{2017 June 14}{S04: Kinematics.}{
% Compute and apply position, velocity, and acceleration vector functions.
% }{2/3}
%
% Suppose the movement of a particle along the curve \(y=x^2\) is described
% by \(\vect r(t)=\<2t-1,4t^2-4t+1\>\). Sketch this curve and plot the point
% \(\vect r(1)\) along with its velocity and acceleration vectors
% \(\vect v(1),\vect a(1)\).

% \exerciseHeader{2017 June 15}{S04: Kinematics.}{
% Compute and apply position, velocity, and acceleration vector functions.
% }{3/3}
%
% Recall that position in ideal projectile motion is given by
% \(\vect r(t) = P_0+\vect{v}_0 t-\frac{1}{2}g\vecj t^2\) where
% \(P_0\) is the initial position, \(\vect{v}_0\) is initial velocity,
% and \(g\) is acceleration due to gravity.
%
% \vspace{1em}
%
% Assume \(g=10\) meters per second squared.
% Prove that a projectile launched from a height of \(60\) meters
% with initial velocity \(\<7,20\>\) meters per second will
% land on the ground after \(6\) seconds and travel a total of
% \(42\) meters horizontally.



% \exerciseHeader{2017 June 15}{C04: VectFuncSTNB.}{
% Compute and apply the arclength parameter and TNB frame for a vector function.
% }{1/4}
%
% Find the arclength parameter \(s(t)\) for the curve given by
% \(\vect r(t)=\<\sin t+\cos t,\sqrt{2}t,\sin t-\cos t\>\). Then give the
% arclength from \(t=0\) to \(t=3\).

% \exerciseHeader{2017 June 16}{C04: VectFuncSTNB.}{
% Compute and apply the arclength parameter and TNB frame for a vector function.
% }{2/4}
%
% Consider the curve parametrized by \(\vect{r}(t)=\<2t,\ln(\sec (2t))\>\) for
% \(-\frac{\pi}{4}\leq t\leq\frac{\pi}{4}\).
% It follows that \(\frac{d\vect r}{dt}=\<2,2\tan (2t)\>\) and
% \(\frac{d\vect T}{dt}=\<-2\sin(2t),2\cos(2t)\>\). Compute the normal vector
% and curvature at the point on this curve where \(t=\frac{\pi}{8}\).

\exerciseHeader{2017 June 19}{C04: VectFuncSTNB.}{
Compute and apply the arclength parameter and TNB frame for a vector function.
}{3/4}

Suppose the unit tangent and normal vectors for a parametrized curve are given
by \(\vect T=\frac{1}{2}\<\cos t-\sin t,\sqrt 2,\cos t+\sin t\>\) and
\(\vect N=\frac{1}{2}\<-\sin t-\cos t,0,-\sin t+\cos t\>\).
Find the binormal vector \(\vect B\) when \(t=\pi\).



% \exerciseHeader{2017 June 16}{S05: MulivarFunc.}{
% Sketch and analyze the domain, level curves, and graph of a two-variable
% real-valued function.
% }{1/3}
%
% Sketch the three level curves for the function \(f(x,y)=x-y^2\) that pass
% through the points \(\<3,2\>\), \(\<1,-1\>\), and \(\<2,1\>\).

\exerciseHeader{2017 June 19}{S05: MulivarFunc.}{
Sketch and analyze the domain, level curves, and graph of a two-variable
real-valued function.
}{2/3}

Graph \(f(x,y)=x^2+y^2\).



\exerciseHeader{2017 June 19}{C05: MulivarCalc.}{
Compute and apply the partial derivatives, gradient, and directional
derivatives of a multivariable real-valued function.
}{1/4}

Find \(\nabla g\) for \(g(x,y,z)=3y^2e^z+4xz\).

\end{document}
