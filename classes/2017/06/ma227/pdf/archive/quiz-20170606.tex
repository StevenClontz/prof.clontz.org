\documentclass[12pt]{article}

\usepackage[letterpaper,margin=1in]{geometry}

\setlength{\parindent}{0pt}

\usepackage{amssymb}
\usepackage{amsmath}

\usepackage{multicol}

\usepackage{tikz}

\newcommand{\headerText}{
  MA 227-103 | Summer 2017 | Dr. Clontz
}

\usepackage{fancyhdr}
\pagestyle{fancy}
\renewcommand{\headrulewidth}{0pt}% Default \headrulewidth is 0.4pt
\renewcommand{\footrulewidth}{0pt}% Default \footrulewidth is 0pt
\chead{\footnotesize\bf\headerText}
\cfoot{}

\newcommand{\csch}{\operatorname{csch}}
\newcommand{\sech}{\operatorname{sech}}

\newcommand{\vect}{\mathbf}
\newcommand{\<}{\left\langle}
\renewcommand{\>}{\right\rangle}
\newcommand{\veci}{\hat{\imath}}
\newcommand{\vecj}{\hat{\jmath}}
\newcommand{\veck}{\hat{k}}

\newcommand{\exerciseHeader}[4]{

  \newpage

  \begin{tikzpicture}[x=1in,y=1in]
    \draw[color=black!20] (0,0) rectangle (4.5,1);
    \draw[color=black!20] (0,0.67) -- (4.5,0.66);
    \draw[color=black!20] (0,0.33) -- (4.5,0.33);

    \node[anchor=west] at (0.1,0.83) {Name:};k
    \node[anchor=west] at (0.1,0.5) {J\#:};
    \node[anchor=west] at (0.1,0.17) {Date: \textbf{#1}};

    \node[anchor=west] at (4.6,0.83) {Exercise Type:};
    \node[anchor=west] at (4.7,0.5) {\textbf{Quiz}};
  \end{tikzpicture}

  \vspace{1em}

  \begin{tikzpicture}[x=1in,y=1in]
    \draw[color=black!50] (0,0) rectangle (6.4,1);
    \draw[color=black!50] (5.4,0) -- (5.4,1);
    \draw[dashed,color=black!20] (5.4,0.25) -- (6.4,0.25);

    \node[anchor=north west,text width=4in,color=black!70] at (0,1) {\footnotesize Standard: This student is able to...};
    \node[anchor=north west,text width=4.5in] at (0.05,0.8) {\textbf{#2} #3};
    \node[anchor=south west,color=black!70] at (0, 0) {\footnotesize #4};

    \node[anchor=north west,color=black!70] at (5.4,0.95) {\footnotesize Mark:};
    \node[anchor=south east,color=black!70] at (5.4,0) {\footnotesize \(\star\) reattempt due on:};
  \end{tikzpicture}

  \vspace{1em}

}



\begin{document}

% \exerciseHeader{2017 June 01}{S01: 3DSpace.}{
% Plot and analyze points and vectors in three-dimensional Euclidean space.
% }{1/3}
%
% Find the magnitude \(\|\vect v\|\) and
% direction \(\frac{1}{\|\vect v\|}\vect v\)
% of the vector \(\vect v = \<2,-4,4\>\).

% \exerciseHeader{2017 June 02}{S01: 3DSpace.}{
% Plot and analyze points and vectors in three-dimensional Euclidean space.
% }{2/3}
%
% Sketch the vector \(\vect v=\<0,2,-3\>\) in \(xyz\) space.
% Then compute \(3\vect v\), and sketch it in \(xyz\) space
% as well.

% \exerciseHeader{2017 June 05}{S01: 3DSpace.}{
% Plot and analyze points and vectors in three-dimensional Euclidean space.
% }{3/3}
%
% In \(xyz\) space,
% sketch the vector \(\vect v=\<1,3,0\>\), the vector \(\vect w\) pointing
% from \(\<1,3,0\>\) to \(\<1,3,4\>\), and the vector \(\vect v+\vect w\).



% \exerciseHeader{2017 June 02}{S02: DotProd.}{
% Compute and apply the dot product of two vectors.
% }{1/3}
%
% Find \(\cos\theta\), where \(\theta\) is the angle between the vectors
% \(\<5,0,-12\>\) and \(\<3,-4,0\>\).

% \exerciseHeader{2017 June 05}{S02: DotProd.}{
% Compute and apply the dot product of two vectors.
% }{2/3}
%
% Find the work done by a force of \(6\) units over a distance of \(4\) units,
% assuming that the force vector is applied at an angle of \(\pi/3\) radians from
% the displacement vector.

\exerciseHeader{2017 June 06}{S02: DotProd.}{
Compute and apply the dot product of two vectors.
}{3/3}

Verify that
\[
  \<3,0,-2\>\cdot(\<-1,2,3\>+\<3,1,4\>)
=
  \<3,0,-2\>\cdot\<-1,2,3\>+\<3,0,-2\>\cdot\<3,1,4\>
\]
by computing both sides separately.



% \exerciseHeader{2017 June 05}{S03: CrossProd.}{
% Compute and apply the cross product of two vectors.
% }{1/3}
%
% Prove that \(\veci\times\veck=-\vecj\), either by computing the cross product
% directly, or by using
% \(\vect{v}\times\vect{w}=(\|\vect v\|\|\vect w\|\sin\theta)\vect{n}\).

\exerciseHeader{2017 June 06}{S03: CrossProd.}{
Compute and apply the cross product of two vectors.
}{2/3}

Prove that \(\<2,-6,4\>\) and \(\<-3,9,-6\>\) are parallel vectors.

\end{document}
